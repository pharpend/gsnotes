\chapter{Euclidean algorithm}
\label{ch:euclid}

The goal of this chapter is to study the \term{Euclidean algorithm},
which is an algorithm for quickly finding the greatest common divisor
of two numbers.

First, we're going to go over some basics, starting with some
inductive proofs about properties of the natural numbers.

\section{Induction}

There are 5 axioms, called the \term{Peano axioms}\footnote{``Peano''
  is pronounced ``pay ahh no''.}, which we can use to define most
properties of the natural numbers.

\begin{aside}
  There are loads of textbooks that prove very basic properties of the
  natural numbers using the Peano axioms. Such properties include
  things like

  \begin{enumerate}
  \item for all natural numbers $a, b, c$

    \begin{zz}
      a + (b + c) = (a + b) + c
    \end{zz}
  \end{enumerate}

  It's not very interesting, but it's great practice if you aren't
  used to proofs. If you are interested in the topic, I would
  recommend Edmund Landau's \xti{Foundations of Analysis}
  \cite{landau}, right after you visit your local mental hospital.
\end{aside}

\begin{definition}[Strong induction]
  \label{def:strong-induction}
  If there is a set $A$ such that
    \begin{itemize}
    \item $1 \in A$,
    \item for every $n$, $\mset{1, 2, \dots, n} \subeq A$ implies
      $S(n) \in A$,
    \end{itemize}

    then every natural number is in $A$.
\end{definition}

The last axiom is called ``strong induction''. There is a weaker, more
common version of the axiom, called ``weak induction''.

\begin{definition}[Weak induction]
  \label{def:weak-induction}
  If there is a set $A$ such that

  \begin{itemize}
  \item $1 \in A$
  \item $n \in A$ implies $S(n) \in A$
  \end{itemize}

  then every natural number $n$ is in $A$.
\end{definition}

Weak induction can be proven from strong induction
(\cref{exc:strong-weak}).\footnote{Generally, the terms ``strong X''
  and ``weak X'' mean that ``weak X'' can be proven from ``strong
  X''.} The concept is usually called \term{mathematical induction},
or just \term{induction}. In general, we apply this in proofs by
having some function $P$, which takes a natural number, and decides if
a given proposition is true or not. If $\scomp{n \in \N}{P(n)}$
satisfies the induction axiom, then we conclude $P(n)$ holds for all
$n$.

\begin{aside}
  Some logic systems don't have the dichotomy of ``true'' or
  ``false'', also called the \term{law of excluded middle}, which
  makes my explanation of induction somewhat reductionist.  If you are
  interested in the topic, check out \link{John Halleck's Logic
    Systems}{http://www.cc.utah.edu/~nahaj/logic/structures/systems/index.html}. \cite{halleck}
\end{aside}

\begin{definition}[Well-ordering principle]
  Every non-empty subset of the natural numbers has a least element.
\end{definition}

\begin{remark}
  The well-ordering principle is equivalent to the principle of
  mathematical induction (\cref{exc:well-ordering}).
\end{remark}

\begin{lemma}
  The sum of the first $n$ positive integers is equal to $n^2$. That
  is,

  \begin{zz}
    n^2 = \sum_{i = 1}^n i
  \end{zz}

  Let's define $S$ to be the set of all natural numbers for which
  $P(n)$ does not hold.
\end{lemma}

% -------------------------------------------------------------------- %
\subsection*{Exercises}

\begin{ExerciseList}
  \Exercise{Prove that strong induction (\cref{def:strong-induction})
    implies weak induction (\cref{def:weak-induction}).}
  \label{exc:strong-weak}

  \Exercise{Using induction, prove that, for all natural numbers
    $n > 2$, there are no integer solutions to

    \begin{zz}
      x^n + y^n = z^n
    \end{zz}
  }
  \label{exc:fermat}

  \Exercise{Prove that the well-ordering principle
    (\cref{def:well-ordering}) is equivalent to mathematical
    induction.}
  \label{exc:well-ordering}
\end{ExerciseList}
\section{Division}

In this section, we are going to derive an algorithm for division with
a remainder, and then prove it's correct.

\begin{definition}[Division algorithm]
  Let $a, b \in \N$, with $a \ge b$. Define

  \begin{zz}
    S = \scomp{n \in \N}{a < nb}
  \end{zz}

  We have $a + 1 \in S$, because $1 \le b$, which implies

  \begin{zz}
    a < a + 1 \le (a + 1)b
  \end{zz}

  Therefore, $S \ne \nil$. By the well-ordering
  principle \parenref{well-ordering}, $S$ contains a smallest element
  $s_0$ such that $a < s_0b$. Therefore, we have

  \begin{zz}
    a \ge \parens{s_0 - 1} b
  \end{zz}

  Let's define $q = s_0 - 1$, $r = a - qb$. We have that $a < (q +
  1)b$, which implies $a < (qb + b)$. We have that $r = a - qb$ (and,
  by extension $r < b$). Therefore, we have $a \ge qb$, and $r = (a -
  qb) \ge 0$.
\end{definition}

Intuitively, we can construct the interval $\clopint{qb, (q + 1)
  b}$. We have that $a$ is somewhere in the interval.

\begin{lemma}[Division with remainder]
  For $a, b \in \Z$, with $b \ne 0$. There exist unique $q, r \in \Z$
  such that $a = qb + r$, with $0 \le r < \abs{b}$.

  \begin{rcl}
    \frac{a}{b} & = & q + \frac{r}{b} \\
    a & = & qb + r \\
  \end{rcl}
\end{lemma}
\begin{proof}
  This can be proved by case analysis.

  \begin{description}
  \item[Case $a, b > 0$] What happens if $b > a$? Put $q = 0$, $r =
    a$, $a = 0\cdot b + a$.
  \item[Case $b < 0 < a$] $a = q(-b)$
  \item[Case $a < 0 < b$] \dots
  \item[Case $a, b < 0$] \dots
  \end{description}

  I couldn't write fast enough to keep up with the notes in class. So,
  my notebook just has an ellipsis in it.
\end{proof}

\begin{example}
  Let's divide $541$ by $130$.

  \begin{zz}
    541 = 4 \cdot 130 + 21
  \end{zz}

  In this case, we have $a = 541$, $b = 130$, $q = 4$, and $r = 21$.
\end{example}

\subsection{Divisibility}

For $b > 0$,

\begin{definition}
 The floor of $x$, denoted $\floor{x}$, is the greatest integer less
 than or equal to $x$.
\end{definition}

\begin{definition}
  $a \mod b$ is the remainder of dividing $a$ by $b$
\end{definition}

\begin{definition}
  All of the following are equivalent:

  \begin{enumerate}
  \item $b$ divides $a$, written $b \div a$
  \item $b$ is a divisor of $a$.
  \item $b$ is a factor of $a$.
  \item $a$ is a multiple of $b$
  \item If $b \ne 0$, then there exists $n \in \Z$ such that $a = bn$
  \item $a = qb + r$ with $r = 0$
  \item $\frac{a}{b} \in \Z$.
  \end{enumerate}
\end{definition}

\begin{example}
  We have $3 \div 39$.

  \begin{rcl}
    39 & = & 3 \cdot 13 \\
    39 & = & 13 \cdot 3 + 0 \\
    \frac{39}{3} & \in & \Z \\
  \end{rcl}
\end{example}

\begin{remark}
  In general, writing things additively and/or multiplicatively is
  better than writing things with division.
\end{remark}

\begin{lemma}
  $a \div 0$ for all $a \in \Z \setminus \mset{0}$.
\end{lemma}
\begin{proof}
  $0 = a \cdot 0$
\end{proof}

\begin{lemma}
  $0 \nmid a$, for all $a \in \Z$
\end{lemma}
\begin{proof}
  By definition.
\end{proof}

\begin{lemma}
  $1 \div a$ for all $a \in \Z$
\end{lemma}
\begin{proof}
  $a = 1 \cdot a$
\end{proof}

\begin{lemma}
  $a \div a$ for all $a \in \Z$
\end{lemma}
\begin{proof}
  $a = a \cdot 1$
\end{proof}

\begin{lemma}
  $b \div a$ iff $-b \div a$ iff $b \div -a$ iff $-b \div -a$
\end{lemma}
\begin{proof}
  \begin{rclmath}
    a & = & bn \\
      & = & (-b)(-n) \\
    (-a) & = & b(-n) \\
         & = & (-b)n \\
  \end{rclmath}
\end{proof}

\begin{proposition}
  $a \div b$ and $b \div c$ implies $a \div c$
\end{proposition}
\begin{proof}
  If $a \div b$, then $b = am$, for some $m \in \Z$. $c = mb = m(na) =
  (mn)a$. Therefore, $a \div c$
\end{proof}

\begin{proposition}
  $a \div b$, with $b \ne 0$ implies $\abs{a} \le \abs{b}$.

  \begin{proof}
    $a \div b$ implies $b = an$, for some $n \in Z$. $b \ne 0$ implies
    $n \ne 0$. $\abs{b} = \abs{a}\abs{n} \ge \abs{a}\cdot 1$, because
    $n \in \Z$.
  \end{proof}
\end{proposition}

\subsection{Divisors}

\begin{definition}
  The \term{divisors} of an integer $a$ are the set of integers $b$
  such that $b \div a$. That is,

  \begin{zz}
    \scomp{b \in \Z}{b \div a}
  \end{zz}
\end{definition}

\begin{remark}
  ``The divisors'' will almost always refer to the positive divisors.
\end{remark}

\begin{example}
  The divisors of $30$ are $\mset{1,2,3,4,6,10,15}$
\end{example}

\begin{example}
  The divisors of $42$ are $\mset{1,2,3,6,7,14,21}$
\end{example}

\begin{remark}
  If $a = bn$, then both $b$ and $n$ are divisors of $a$.
\end{remark}

\begin{definition}
  The \term{common divisors} of two nonzero integers $a$ and $b$ are
  the numbers that divide both $a$ and $b$.
\end{definition}

\begin{example}
  The common divisors of $30$ and $42$ are $\mset{1,2,3,6}$.
\end{example}

\begin{definition}
  \label{def:gcd}
  The \term{greatest common divisor} of two nonzero integers $a$ and
  $b$ is the largest integer that divides both $a$ and $b$. That is,

  \begin{zz}
    \gcdp{a,b} = \max \scomp{d \in \Z}{d \div a \land d \div b}
  \end{zz}

  Equivalently, 

  \begin{itemize}
  \item $\gcdp{a,b} \div a$
  \item $\gcdp{a,b} \div b$
  \item $n \div a$ and $n \div b$ implies $n \le \gcdp{a, b}$
  \end{itemize}
\end{definition}

\begin{definition}
  A \term{linear combination} of $a$ and $b$ is an integer $L \in \Z$
  such that $ax + by = L$, where $x, y \in \Z$
\end{definition}

\begin{lemma}
  \label{thm:common-div-linear-combo}
  \label{thm:linear-combo-divisors}
  \label{linear-combo-divisors}
  Any common divisor of $a$ and $b$ is a divisor of any linear
  combination thereof.
\end{lemma}
\begin{proof}
  Let $L = ax + by$ be a linear combination of $a$ and $b$. We have
  that $d \div a$, and therefore $a = dm$. We also have $d \div b$,
  and therefore $b = dn$. Therefore,

  \begin{rcl}
    L & = & (dm)x + (dn)y \\
      & = & d (mx + ny) \\
  \end{rcl}

  Therefore, $d \div L$.
\end{proof}

\begin{corollary}
  Given $a = qb + r$, $\gcdp{a,b} = \gcdp{b,r}$.
\end{corollary}
\begin{proof}
  $a = qb + r$ is a linear combination of $b$ and $r$. By
  \cref{thm:linear-combo-divisors}, all common divisors of $b$ and $r$
  also divide $a$.

  We also have $r = a - qb$, which implies that any common divisor of
  $a$ and $b$ is also a divisor of $r$.  This implies

  \begin{zz}
    \scomp{d \in \Z}{d \div a \land d \div b} = \scomp{d \in \Z}{d
      \div b \land d \div r}
  \end{zz}

  Because the two sets are equal, their maximum elements are
  equal. Therefore,

  \begin{zz}
    \gcdp{a,b} = \gcdp{b,r}
  \end{zz}

  QED.
\end{proof}

\begin{example}
  \label{ex:divisors-linear-combos}
  Remember from before,

  \begin{zz}
    42 = 1 \cdot 30 + 12
  \end{zz}

  The divisors of $42$ are

  \begin{zz}
    \mset{1,2,3,6,7,14,21,42}
  \end{zz}

  The divisors of $30$ are

  \begin{zz}
    \mset{1,2,3,5,6,12,30}
  \end{zz}

  The divisors of $r = 12$ are

  \begin{zz}
    \mset{1,2,3,6,12}
  \end{zz}

  The common divisors of all three are $\mset{1,2,3,6}$. Their
  greatest common divisor is $6$.

  We can write

  \begin{alignmath}{rcccl}
    42 & = & 1 \cdot 30 & + & 12 \\
    30 & = & 2 \cdot 12 & + & 6 \\
    12 & = & 2 \cdot 6 & + & 0 \\
  \end{alignmath}

  Therefore, $6 \div 12$, $6 \div 30$, and $6 \div 42$.
\end{example}

\subsection{Euclidean Algorithm}

The \term{Euclidean algorithm} is a fast method for finding the
greatest common divisor of two numbers. We purposefully stumbled upon
it in \cref{ex:divisors-linear-combos}. Choose positive integers
$a, b$

\begin{alignmath}{rclcrcl}
  a & = & q_1b + r_1 & \text{where} & 0 \le r_1 < b \\
  b & = & q_2r_1 + r_2 & \text{where} & 0 \le r_2 < r_1 \\
  r_1 & = & q_3r_2 + r_3 & \text{where} & 0 \le r_3 < r_2 \\
  r_2 & = & q_4r_3 + r_4 & \text{where} & 0 \le r_4 < r_3 \\
      &   &              & \dots \\
  r_{n - 2} & = & q_n r_{n - 1} + r_n & \text{where} & 0 \le r_n < r_{n - 1} \\
  r_{n - 1} & = & q_{n + 1}r_n + 0
\end{alignmath}

The last nonzero remainder, $r_n$, is the greatest common divisor.
The fact that the algorithm terminates follows from the fact that the
set $b > r_1 > r_2 > \dots > r_n$ must have a least element (by the
well-ordering principle), which implies $r_{n + 1} = 0$.

\begin{example}
  Let's use the Euclidean algorithm to find $\gcdp{541,130}$

  \begin{alignmath}{rcrcl}
    541 & = & 4 \cdot 130 & + & 21 \\
    130 & = & 6 \cdot 21 & + & 4 \\
    21 & = & 5 \cdot 4 & + & 1 \\
    4 & = & 4 \cdot 1 & + & 0 \\
  \end{alignmath}

  Since $1$ is the last nonzero remainder, it's the greatest common
  divisor.
\end{example}

\begin{remark}
  Aaron was tricky. $541$ is a prime number, which implies its only
  divisors are $1$ and $541$. Therefore, its greatest common divisor
  with any number absolutely less than $541$ will obviously be $1$.
\end{remark}

\begin{example}
  Let's find $\gcdp{10469,10013}$ using the Euclidean algorithm.

  \begin{alignmath}{rcrcl}
    10469 & = & 1 \cdot 10013 & + & 456 \\
    10013 & = & 21 \cdot 456 & + & 437 \\
    456 & = & 1 \cdot 437 & + & 19 \\
    437 & = & 23 \cdot 19 & + & 0 \\
  \end{alignmath}

  Therefore, $\gcdp{10469,10013} = 19$
\end{example}

\begin{example}
  Find $\gcdp{2717,2431}$ using the Euclidean algorithm

  \begin{alignmath}{rcrcl}
    2717 & = & 1 \cdot 2431 & + & 286 \\
         & \dots &
  \end{alignmath}
\end{example}

\begin{remark}
  Turns out, worst case, the Euclidean algorithm terminates in
  $5 \cdot \log_{10} r$ steps.
\end{remark}

\section{Euclidean Algorithm}

The \term{Euclidean algorithm} is a fast method for finding the
greatest common divisor of two numbers. We purposefully stumbled upon
it in \cref{ex:divisors-linear-combos}. Choose positive integers
$a, b$

\begin{alignmath}{rclcrcl}
  a & = & q_1b + r_1 & \text{where} & 0 \le r_1 < b \\
  b & = & q_2r_1 + r_2 & \text{where} & 0 \le r_2 < r_1 \\
  r_1 & = & q_3r_2 + r_3 & \text{where} & 0 \le r_3 < r_2 \\
  r_2 & = & q_4r_3 + r_4 & \text{where} & 0 \le r_4 < r_3 \\
      &   &              & \dots \\
  r_{n - 2} & = & q_n r_{n - 1} + r_n & \text{where} & 0 \le r_n < r_{n - 1} \\
  r_{n - 1} & = & q_{n + 1}r_n + 0
\end{alignmath}

The last nonzero remainder, $r_n$, is the greatest common divisor.
The fact that the algorithm terminates follows from the fact that the
set $b > r_1 > r_2 > \dots > r_n$ must have a least element (by the
well-ordering principle), which implies $r_{n + 1} = 0$.

\begin{example}
  Let's use the Euclidean algorithm to find $\gcdp{541,130}$

  \begin{alignmath}{rcrcl}
    541 & = & 4 \cdot 130 & + & 21 \\
    130 & = & 6 \cdot 21 & + & 4 \\
    21 & = & 5 \cdot 4 & + & 1 \\
    4 & = & 4 \cdot 1 & + & 0 \\
  \end{alignmath}

  Since $1$ is the last nonzero remainder, it's the greatest common
  divisor.
\end{example}

\begin{remark}
  Aaron was tricky. $541$ is a prime number, which implies its only
  divisors are $1$ and $541$. Therefore, its greatest common divisor
  with any number absolutely less than $541$ will obviously be $1$.
\end{remark}

\begin{example}
  Let's find $\gcdp{10469,10013}$ using the Euclidean algorithm.

  \begin{alignmath}{rcrcl}
    10469 & = & 1 \cdot 10013 & + & 456 \\
    10013 & = & 21 \cdot 456 & + & 437 \\
    456 & = & 1 \cdot 437 & + & 19 \\
    437 & = & 23 \cdot 19 & + & 0 \\
  \end{alignmath}

  Therefore, $\gcdp{10469,10013} = 19$
\end{example}

\begin{example}
  Find $\gcdp{2717,2431}$ using the Euclidean algorithm

  \begin{alignmath}{rcrcl}
    2717 & = & 1 \cdot 2431 & + & 286 \\
         & \dots &
  \end{alignmath}
\end{example}

\begin{remark}
  Turns out, worst case, the Euclidean algorithm terminates in
  $5 \cdot \log_{10} r$ steps.
\end{remark}

% \section{Bezout's lemma}

Savin calls this the ``Fundamental Theorem of Arithmetic''. That name
is typically used to refer to another theorem, so we'll use ``Bezout's
lemma'' for now.



\section*{Exercises}

\subsection*{Induction}

\begin{exercise}
  \label{exc:strong-weak}
  Prove that strong induction \parenref{def:strong-induction}
  implies weak induction \parenref{def:weak-induction}.
\end{exercise}

\begin{exercise}
  \label{exc:well-ordering}
  Prove that the well-ordering
  principle \parenref{def:well-ordering} is equivalent to
  mathematical induction \parenref{def:strong-induction}.
\end{exercise}

\exercise{Prove \cref{thm:n-positive-ints} without induction.}

\begin{exercise}
  \label{exc:fermat}
  \label{fermat}
  Using induction, prove that, for all natural numbers $n > 2$,
  there are no integer solutions to

  \begin{zz}
    x^n + y^n = z^n
  \end{zz}
\end{exercise}

\subsubsection*{Aaron's Induction Exercises}

\begin{exercise}
  \label{aaron-induction-1}
  Using induction, prove each of the following

  \begin{enumerate}[label=(\alph*)]
  \item For $n \ge 1$, the sum of the first $n$ positive integers is
    equal to $\frac{n(n + 1)}{2}$. That is,

    \begin{zz}
      1 + 2 + 3 + \dots + n = \frac{n(n + 1)}{2}
    \end{zz}
  \item For $n \ge 1$, sum of the first $n$ perfect squares is equal
    to $\frac{n(n + 1)(2n + 1)}{6}$. That is,

    \begin{zz}
      1^2 + 2^2 + 3^2 + \dots + n^2 = \frac{n (n + 1)(2n + 1)}{6}
    \end{zz}
  \item Using induction, prove that for $n \ge 1$, the sum of the
    first $n$ positive cubes is equal to the square of the sum of
    the first $n$ positive integers. That is,

    \begin{zz}
      1^3 + 2^3 + 3^3 + \dots + n^3 = \parens{1 + 2 + 3 + \dots + n}^2
    \end{zz}

    Hint: use part (a).
  \item Fix a real number $x \ge -1$. For any $n \ge 1$,

    \begin{zz}
      (1 + x)^n \ge 1 + nx
    \end{zz}

  \item For $n \ge 1$,

    \begin{zz}
      1(2) + 2(3) + 3(4) + \dots + n(n + 1) = \frac{n(n+1)(n+2)}{3}
    \end{zz}
  \item For $n \ge -1$, the integer $2n^3 + 7n$ is divisible by $3$.
  \item For $n \ge 1$, the integer $n^2 + (n + 2)^2 + (n + 4)^2$ is
    divisible by $12$.
  \end{enumerate}
\end{exercise}

\begin{exercise}
  Compute the first several values of the following sequences, and
  look for a pattern. Guess the formula for the $n$th term. Prove your
  guess is correct using induction.

  \begin{enumerate}[label=(\alph*)]
  \item $a_n = \sum_{k = 1}^n 2^{-k}$
  \item $b_n = \sum_{k = 1}^n \frac{1}{k(k + 1)}$
  \end{enumerate}
\end{exercise}

\begin{exercise}
  The Fibonacci sequence $F_n$ is defined recursively by

  \begin{rcl}
    F_0 & = & 0 \\
    F_1 & = & 1 \\
    F_n & = & F_{n - 1} + F_{n - 2} \\
  \end{rcl}

  \begin{enumerate}[label=(\alph*)]
  \item Calculate $F_n$ for $2 \le n \le 15$.
  \end{enumerate}

  Use induction to prove the following statements

  \begin{enumerate}[label=(\alph*),start=2]
  \item For $n \ge 0$,

    \begin{zz}
      F_1 + F_2 + F_3 + \dots + F_n = F_{n + 2} - 1
    \end{zz}

  \item For $n \ge 1$,

    \begin{zz}
      F_1 + F_3 + F_5 + \dots + F_{2n - 1} = F_{2n}
    \end{zz}

  \item for $n \ge 1$,

    \begin{zz}
      F_2 + F_4 + F_6 + \dots + F_{2n} = F_{2n + 1} - 1
    \end{zz}

  \item Use induction to prove that for $n \ge 0$,

    \begin{zz}
      \parens{F_{n + 1}}^2 - F_nF_{n + 2} = (-1)^n
    \end{zz}
    
  \end{enumerate}
\end{exercise}

\subsection*{Division}

\subsubsection*{Aaron's Divisibility exercises}

\begin{exercise}
  Prove the following statements about the integers $a, b, c, d$

  \begin{enumerate}[label=(\alph*)]
  \item If $a \div 1$, then $a = \pm 1$.
  \item If $a \div b$, and $b \div a$, then $a = \pm b$
  \item If $a \div b$, and $c \div d$, then $ab \div cd$
  \item If $a \div b$, and $a \div c$, then $a^2 \div bc$
  \item If $a \div b$, and $c \ne 0$, then $ac \div bc$
  \item If $ac \div bc$, and $c \ne 0$, then $a \div b$
  \end{enumerate}
\end{exercise}

\begin{exercise}
  Disprove the following statements about positive integers $a$, $b$,
  and $c$ by finding explicit counterexamples:

  \begin{enumerate}[label=(\alph*)]
  \item If $a \div (b + c)$, then $a \div b$ and $a \div c$.
  \item If $a \div bc$, then $a \div b$ or $a \div c$.
  \end{enumerate}
\end{exercise}

\begin{exercise}
  Prove the following statements

  \begin{enumerate}[label=(\alph*)]
  \item For any integer $n$, exactly one of $n$, $n + 2$, or $n + 4$
    is divisible by $3$
  \item For any integer $n$, the integer $2n^3 + 7n$ is divisible by
    $3$.
  \item The square of any even integer is divisible by $4$.
  \item The square of any odd integer is of the form $8n + 1$.
  \item For any odd integer $n$, the integer
    $n^2 + (n + 2)^2 + (n + 4)^2 + 1$ is divisible by $12$.
  \item For any integer $n$, the integer $(n + 1)^3 - n^3$ is odd.
  \end{enumerate}
\end{exercise}

\begin{exercise}
  \label{exc:gcds}
  Use the Euclidean algorithm to find the greatest common divisors
  below.

  \begin{enumerate}[label=(\alph*)]
  \item $\gcdp{339, 82}$
  \item $\gcdp{8415, 1078}$
  \item $\gcdp{1989, 1881}$
  \end{enumerate}
\end{exercise}

\begin{exercise}
  \label{exc:cfs}
  Use the continued fraction algorithm to write the following rational
  numbers as continued fractions. Then, compute the second-to-last
  convergent of the continued fraction.

  \begin{enumerate}[label=(\alph*)]
  \item $\frac{339}{82}$
  \item $\frac{8415}{1078}$
  \item $\frac{1989}{1881}$
  \end{enumerate}
\end{exercise}

\begin{exercise}
  \label{exc:intsolutions}
  Use the Euclidean algorithm output from \cref{exc:gcds} to find
  integer solutions to the following equations. Compare these
  solutions to the second-to-last convergents calculated in
  \cref{exc:cfs}.

  \begin{enumerate}[label=(\alph*)]
  \item $339x + 82y = \gcdp{339, 82}$
  \item $8415x + 1078y = \gcdp{8415, 1078}$
  \item $1989x + 1881y = \gcdp{1989, 1881}$
  \end{enumerate}
\end{exercise}

\begin{exercise}
  Find \xtb{all} integer solutions to the equations in
  \cref{exc:intsolutions}.
\end{exercise}