\section{Euclidean Algorithm}

The \term{Euclidean algorithm} is a fast method for finding the
greatest common divisor of two numbers. We purposefully stumbled upon
it in \cref{ex:divisors-linear-combos}. Choose positive integers
$a, b$

\begin{alignmath}{rclcrcl}
  a & = & q_1b + r_1 & \text{where} & 0 \le r_1 < b \\
  b & = & q_2r_1 + r_2 & \text{where} & 0 \le r_2 < r_1 \\
  r_1 & = & q_3r_2 + r_3 & \text{where} & 0 \le r_3 < r_2 \\
  r_2 & = & q_4r_3 + r_4 & \text{where} & 0 \le r_4 < r_3 \\
      &   &              & \dots \\
  r_{n - 2} & = & q_n r_{n - 1} + r_n & \text{where} & 0 \le r_n < r_{n - 1} \\
  r_{n - 1} & = & q_{n + 1}r_n + 0
\end{alignmath}

The last nonzero remainder, $r_n$, is the greatest common divisor.
The fact that the algorithm terminates follows from the fact that the
set $b > r_1 > r_2 > \dots > r_n$ must have a least element (by the
well-ordering principle), which implies $r_{n + 1} = 0$.

\begin{example}
  Let's use the Euclidean algorithm to find $\gcdp{541,130}$

  \begin{alignmath}{rcrcl}
    541 & = & 4 \cdot 130 & + & 21 \\
    130 & = & 6 \cdot 21 & + & 4 \\
    21 & = & 5 \cdot 4 & + & 1 \\
    4 & = & 4 \cdot 1 & + & 0 \\
  \end{alignmath}

  Since $1$ is the last nonzero remainder, it's the greatest common
  divisor.
\end{example}

\begin{remark}
  Aaron was tricky. $541$ is a prime number, which implies its only
  divisors are $1$ and $541$. Therefore, its greatest common divisor
  with any number absolutely less than $541$ will obviously be $1$.
\end{remark}

\begin{example}
  Let's find $\gcdp{10469,10013}$ using the Euclidean algorithm.

  \begin{alignmath}{rcrcl}
    10469 & = & 1 \cdot 10013 & + & 456 \\
    10013 & = & 21 \cdot 456 & + & 437 \\
    456 & = & 1 \cdot 437 & + & 19 \\
    437 & = & 23 \cdot 19 & + & 0 \\
  \end{alignmath}

  Therefore, $\gcdp{10469,10013} = 19$
\end{example}

\begin{example}
  Find $\gcdp{2717,2431}$ using the Euclidean algorithm

  \begin{alignmath}{rcrcl}
    2717 & = & 1 \cdot 2431 & + & 286 \\
         & \dots &
  \end{alignmath}
\end{example}

\begin{remark}
  Turns out, worst case, the Euclidean algorithm terminates in
  $5 \cdot \log_{10} r$ steps.
\end{remark}
