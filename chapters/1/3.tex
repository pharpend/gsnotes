\section{Bezout's lemma}

Savin calls this the ``Fundamental Theorem of Arithmetic''. That name
is typically used to refer to another theorem, so we'll use ``Bezout's
lemma'' for now. We'll use it in the future to prove the uniqueness of
factorization.

\begin{lemma}[Bezout's lemma]
  \label{bezout}
  For any $a, b \in \Z$, there exist $x, y \in \Z$ such that

  \begin{zz}
    ax + by = \gcdp{a,b}
  \end{zz}
\end{lemma}

There are two proofs.

\begin{proof}[Existence proof]
  In the case where $a$ is zero, we note that
  $0 \cdot x + b \cdot 1 = b = \gcdp{0, b}$, for all $x \in \Z$. An
  analogous argument applies when $b = 0$, or when both $a$ and $b$
  are zero.

  In the case where both $a$ and $b$, are nonzero, let $L$ be the set
  of integer linear combinations of $a$ and $b$. That is,

  \begin{zz}
    L = \scomp{ax + by \in \Z}{x,y \in \Z}
  \end{zz}

  To get the positive integer linear combinations, we intersect $L$
  with $\N$ to get

  \begin{zz}
    L \cap \N = \scomp{ax + by \in \N}{x,y \in \Z}
  \end{zz}

  This set is nonempty, because it contains both $\abs{a}$ and
  $\abs{b}$ as elements. Define $\ell_0 = \fun{\min}{L \cap \N}$. By
  the well-ordering principle \parenref{well-ordering}, there exists
  $x_0, y_0 \in \Z$ such that

  \begin{zz}
    \ell_0 = ax_0 + by_0
  \end{zz}

  We want to prove that $\ell_0$ is the greatest common divisor. We'll
  abbreviate $\gcdp{a,b} = d$ from now on.

  \begin{enumerate}
  \item We have $\ell_0$ is a linear combination of $a$ and $b$. Any
    common divisor of $a$ and $b$ divides any linear combination
    thereof \parenref{linear-combo-divisors}. Therefore, $d$ divides
    $\ell_0$, which implies $d \le \ell_0$.
  \item Going to use the division algorithm to write $q = a\ell_0 +
    r$.

    \begin{rcl}
      r & = & a - q\ell_0 \\
        & = & a - q\parens{ax_0 + by_0} \\
        & = & a\parens{1 - qx_0} + b\parens{qy_0} \\
    \end{rcl}

    This is a linear combination of $a$ and $b$, which puts $r \in
    L$. We have by definition of the division algorithm
    $0 \le r < \ell_0$, and previously that $\ell_0$ is the smallest
    positive element in $L$. Therefore $r = 0$. Therefore,
    $\ell_0 \div a$. A similar argument shows $\ell_0 \div
    b$. Therefore, $\ell_0$ is a common divisor of $a$ and $b$. We
    previously had that $d \le \ell_0$, and $d$ is the greatest common
    divisor. Therefore $\ell_0 = d$.
  \end{enumerate}
\end{proof}

This is merely an existence proof. It doesn't give us a way to find
the linear combinations. We will give such a proof right here.

\begin{proof}[Constructive proof]
  For simplicity, let's say the Euclidean Algorithm terminates in $4$
  steps. Therefore,

  \begin{rcl}
    r_1 & = & a - q_1 b \\
    r_2 & = & b - q_2 r_1 \\
    r_3 & = & r_1 - q_3 r_2 \\
        & = & r_1 - q_3 \parens{b - q_2 r_1} \\
        & = & r_1 - q_3 \brackets{b - q_2 \parens{a - q_1b}} \\
        & = & a\parens{\dots} + b\parens{\dots}
  \end{rcl}

  Therefore, we have $r_3$ as an explicit linear combination of $a$
  and $b$.
\end{proof}

\begin{example}
  Applying the Euclidean algorithm to $10469$ and $10013$, we get

  \begin{alignmath}{rcrcl}
    10469 & = & 1 \cdot 10013 & + & 456 \\
    10013 & = & 21 \cdot 456 & + & 437 \\
    456 & = & 1 \cdot 437 & + & 19 \\
    437 & = & 23 \cdot 19 & + & 0 \\
  \end{alignmath}

  If we rewrite it in our remainder form, we get

  \begin{alignmath}{rcrcl}
    19 & = & 456 & - & 437 \\
       & = & 456 & - & \parens{10013 - 21\cdot 456} \\
       & = & -10013 & + & 22 \cdot 456 \\
       & = & -10013 & + & 22 \cdot (10469 - 10013) \\
       & = & 22 \cdot 10469 & - & 23 \cdot 10013 \\
       & \dots & 
  \end{alignmath}

  Therefore, we have a solution for
  $10469x + 10013y = \gcdp{10469,10013}$.
\end{example}

\begin{lemma}[Euclid's Lemma]
  \label{euclids-lemma}
  If $c \div ab$ and $\gcdp{a,c} = 1$, then $c \div b$.
\end{lemma}
\begin{proof}
  We use Bezout's lemma to write $ax + cy = 1$ for some $x, y \in
  \Z$. We multiply by $b$ to get $abx + bcy = b$. We have $c \div
  ab$ and (trivially) $c \div bc$. Therefore, $c \div abx + bcy$, if
  and only if $c \div b$.
\end{proof}

\begin{lemma}
  If $r \ne 0$, then $\gcdp{ra,rb} = \abs{r}\gcdp{a,b}$.
\end{lemma}
\begin{proof}
  We simply use Bezout's lemma twice. Put $d = \gcdp{a,b}$, and $D =
  \gcdp{ra,rb}$. Our goal is to prove $d = rD$. By Bezout's lemma, we
  have $d = ax + by$. Therefore, $rd = rax + rby$, which is a linear
  combination of $ra$ and $rb$. By \cref{linear-combo-divisors}, we
  have $D \div rd$.

  Applying Bezout's lemma again, we get $D = rau + rbv$. Trivially, we
  have $d \div a \implies rd \div ra$ and $d \div b \implies rd \div
  rb$. Therefore, $rd$ is a common div of $ra$ and $rb$, which implies
  $rd \div D$.
\end{proof}

\begin{definition}
  Two integers are \term{relatively prime} if their greatest common
  divisor is $1$.
\end{definition}

\begin{corollary}
  If $d = \gcdp{a,b}$, then $\gcdp{\frac{a}{d}, \frac{b}{d}} = 1$.
\end{corollary}

\begin{aside}
  A large set of number theory problems are \term{Diophantine
    equations}. Diophantus was some dude circa approximately 200
  CE. We want integer solutions to equations like

  \begin{itemize}
  \item $a^2 + b^2 = c^2$, called \term{Pythagorean triples}, where
    there are infinitely many
  \item $a^n + b^n = c^n$ for $n \ge 2$, where there are
    none \parenref{fermat}.
  \item $y^2 = x^3 + ax + b$, called \term{Elliptic curves}. These are
    very difficult to solve, so some cryptography protocols utilize them.
  \end{itemize}
\end{aside}

\begin{lemma}
  If $d = \gcdp{a,b}$, and $d \ndiv c$, then there are no solutions to
  \begin{zz}
    ax + by = c
  \end{zz}
\end{lemma}
\begin{proof}
  If there is a solution, then $c$ is a linear combination of $a$ and
  $b$, and therefore $d \div c$, a contradiction.
\end{proof}

\begin{lemma}
  If $d \div c$, then there is a solution to $ax + by = c$.
\end{lemma}
\begin{proof}
  We use Bezout's lemma to get $ax + by = d$. If $d \div c$, then
  $c = dn$ for some $n \in \Z$. Therefore, we can write
  $anx + bny = dn = c$.
\end{proof}

\begin{lemma}
  If $\mv{x_0, y_0}$ is a solution to $ax + by = c$, then so is
  $\mv{x_0 + \frac{b}{d} t, y_0 - \frac{a}{d}t}$ for any $t \in \Z$.
\end{lemma}
\begin{proof}
  Take $a \parens{x_0 + \frac{b}{d}t} + b \parens{y_0 -
    \frac{a}{d}t}$. This is equal to

  \begin{zz}
    ax_0 + by_0 + \frac{ab}{d} t - \frac{ba}{d}t
  \end{zz}

  Which is equal to $c$.
\end{proof}

\begin{lemma}
  This accounts for \emph{all} solutions. That is, if $\mv{x_0, y_0}$
  and $\mv{x_1,y_1}$ are solutions, then there exists $t \in \Z$ such
  that
  \begin{rcl}
    x_1 & = & x_0 + \frac{b}{d} t \\
    y_1 & = & y_0 - \frac{a}{d} t
  \end{rcl}
\end{lemma}
\begin{proof}
  We know
  \begin{rcl}
    ax_0 + by_0 & = & c \\
    ax_1 + by_1 & = &   \\
    a(x_0 - x_1) + b(y_0 - y_1) & = & 0 \\
    a(x_0 - x_1) & = & b(y_1 - y_0) \\
    \frac{a}{d}(x_0 - x_1) & = & \frac{b}{d}(y_1 - y_0) \\
  \end{rcl}

  This is still an equation in $\Z$, which implies $\frac{a}{d} \div
  \frac{b}{d}\parens{y_1 - y_0}$. With $\gcdp{\frac{a}{d},\frac{b}{d}}
  = 1$, Euclid's lemma implies $\frac{a}{d} \div \parens{y_1 -
    y_0}$. Therefore, $y_1 - y_0 = \frac{a}{d}(-t)$ where $-t \in
  \Z$. Now, we have $y_1 = y_0 - \frac{a}{d} t$, $a\parens{x_0 - x_1}
  = b\parenfrac{a}{d}(-t)$.

  By the way, I should have mentioned this earlier, we're assuming
  $a \ne 0$. We cancel the $a$s to get $x_1 = x_0 + \parens{b}{d}t$.
\end{proof}

\begin{theorem}
  The equation $ax + by = c$ has a solution $x, y \in \Z$ if and only
  if $\gcdp{a,b} \div c$. Moreover, if $\mv{x_0, y_0}$is any soluton,
  then the set
  \begin{zz}
    \scomp{\mv{x_0 + \frac{b}{d}t, y_0 - \frac{a}{d}t} \in \Z^2}{t \in \Z}
  \end{zz}
  is the set of integer solutions.
\end{theorem}

We'll prove this later.

\begin{example}[Zhang Quijian's chicken problem]
  If a rooster is worth five coins, a hen is worth three coins, and
  one coin is worth three chicks, how can you buy $100$ animals for
  $100$ coins?

  \begin{solution}
    
  \end{solution}
\end{example}