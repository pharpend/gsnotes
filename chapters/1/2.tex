\section{Division}

In this section, we are going to derive an algorithm for division with
a remainder, and then prove it's correct.

\begin{definition}[Division algorithm]
  Let $a, b \in \N$, with $a \ge b$. Define

  \begin{zz}
    S = \scomp{n \in \N}{a < nb}
  \end{zz}

  We have $a + 1 \in S$, because $1 \le b$, which implies

  \begin{zz}
    a < a + 1 \le (a + 1)b
  \end{zz}

  Therefore, $S \ne \nil$. By the well-ordering
  principle \parenref{well-ordering}, $S$ contains a smallest element
  $s_0$ such that $a < s_0b$. Therefore, we have

  \begin{zz}
    a \ge \parens{s_0 - 1} b
  \end{zz}

  Let's define $q = s_0 - 1$, $r = a - qb$. We have that $a < (q +
  1)b$, which implies $a < (qb + b)$. We have that $r = a - qb$ (and,
  by extension $r < b$). Therefore, we have $a \ge qb$, and $r = (a -
  qb) \ge 0$.
\end{definition}

Intuitively, we can construct the interval $\clopint{qb, (q + 1)
  b}$. We have that $a$ is somewhere in the interval.

\begin{lemma}[Division with remainder]
  For $a, b \in \Z$, with $b \ne 0$. There exist unique $q, r \in \Z$
  such that $a = qb + r$, with $0 \le r < \abs{b}$.

  \begin{rcl}
    \frac{a}{b} & = & q + \frac{r}{b} \\
    a & = & qb + r \\
  \end{rcl}
\end{lemma}
\begin{proof}
  This can be proved by case analysis.

  \begin{description}
  \item[Case $a, b > 0$] What happens if $b > a$? Put $q = 0$, $r =
    a$, $a = 0\cdot b + a$.
  \item[Case $b < 0 < a$] $a = q(-b)$
  \item[Case $a < 0 < b$] \dots
  \item[Case $a, b < 0$] \dots
  \end{description}

  I couldn't write fast enough to keep up with the notes in class. So,
  my notebook just has an ellipsis in it.
\end{proof}

\begin{example}
  Let's divide $541$ by $130$.

  \begin{zz}
    541 = 4 \cdot 130 + 21
  \end{zz}

  In this case, we have $a = 541$, $b = 130$, $q = 4$, and $r = 21$.
\end{example}

\subsection{Divisibility}

For $b > 0$,

\begin{definition}
 The floor of $x$, denoted $\floor{x}$, is the greatest integer less
 than or equal to $x$.
\end{definition}

\begin{definition}
  $a \mod b$ is the remainder of dividing $a$ by $b$
\end{definition}

\begin{definition}
  All of the following are equivalent:

  \begin{enumerate}
  \item $b$ divides $a$, written $b \div a$
  \item $b$ is a divisor of $a$.
  \item $b$ is a factor of $a$.
  \item $a$ is a multiple of $b$
  \item If $b \ne 0$, then there exists $n \in \Z$ such that $a = bn$
  \item $a = qb + r$ with $r = 0$
  \item $\frac{a}{b} \in \Z$.
  \end{enumerate}
\end{definition}

\begin{example}
  We have $3 \div 39$.

  \begin{rcl}
    39 & = & 3 \cdot 13 \\
    39 & = & 13 \cdot 3 + 0 \\
    \frac{39}{3} & \in & \Z \\
  \end{rcl}
\end{example}

\begin{remark}
  In general, writing things additively and/or multiplicatively is
  better than writing things with division.
\end{remark}

\begin{lemma}
  $a \div 0$ for all $a \in \Z \setminus \mset{0}$.
\end{lemma}
\begin{proof}
  $0 = a \cdot 0$
\end{proof}

\begin{lemma}
  $0 \nmid a$, for all $a \in \Z$
\end{lemma}
\begin{proof}
  By definition.
\end{proof}

\begin{lemma}
  $1 \div a$ for all $a \in \Z$
\end{lemma}
\begin{proof}
  $a = 1 \cdot a$
\end{proof}

\begin{lemma}
  $a \div a$ for all $a \in \Z$
\end{lemma}
\begin{proof}
  $a = a \cdot 1$
\end{proof}

\begin{lemma}
  $b \div a$ iff $-b \div a$ iff $b \div -a$ iff $-b \div -a$
\end{lemma}
\begin{proof}
  \begin{rclmath}
    a & = & bn \\
      & = & (-b)(-n) \\
    (-a) & = & b(-n) \\
         & = & (-b)n \\
  \end{rclmath}
\end{proof}

\begin{proposition}
  $a \div b$ and $b \div c$ implies $a \div c$
\end{proposition}
\begin{proof}
  If $a \div b$, then $b = am$, for some $m \in \Z$. $c = mb = m(na) =
  (mn)a$. Therefore, $a \div c$
\end{proof}

\begin{proposition}
  $a \div b$, with $b \ne 0$ implies $\abs{a} \le \abs{b}$.

  \begin{proof}
    $a \div b$ implies $b = an$, for some $n \in Z$. $b \ne 0$ implies
    $n \ne 0$. $\abs{b} = \abs{a}\abs{n} \ge \abs{a}\cdot 1$, because
    $n \in \Z$.
  \end{proof}
\end{proposition}

\subsection{Divisors}

\begin{definition}
  The \term{divisors} of an integer $a$ are the set of integers $b$
  such that $b \div a$. That is,

  \begin{zz}
    \scomp{b \in \Z}{b \div a}
  \end{zz}
\end{definition}

\begin{remark}
  ``The divisors'' will almost always refer to the positive divisors.
\end{remark}

\begin{example}
  The divisors of $30$ are $\mset{1,2,3,4,6,10,15}$
\end{example}

\begin{example}
  The divisors of $42$ are $\mset{1,2,3,6,7,14,21}$
\end{example}

\begin{remark}
  If $a = bn$, then both $b$ and $n$ are divisors of $a$.
\end{remark}

\begin{definition}
  The \term{common divisors} of two nonzero integers $a$ and $b$ are
  the numbers that divide both $a$ and $b$.
\end{definition}

\begin{example}
  The common divisors of $30$ and $42$ are $\mset{1,2,3,6}$.
\end{example}

\begin{definition}
  \label{def:gcd}
  The \term{greatest common divisor} of two nonzero integers $a$ and
  $b$ is the largest integer that divides both $a$ and $b$. That is,

  \begin{zz}
    \gcdp{a,b} = \max \scomp{d \in \Z}{d \div a \land d \div b}
  \end{zz}

  Equivalently, 

  \begin{itemize}
  \item $\gcdp{a,b} \div a$
  \item $\gcdp{a,b} \div b$
  \item $n \div a$ and $n \div b$ implies $n \le \gcdp{a, b}$
  \end{itemize}
\end{definition}

\begin{definition}
  A \term{linear combination} of $a$ and $b$ is an integer $L \in \Z$
  such that $ax + by = L$, where $x, y \in \Z$
\end{definition}

\begin{lemma}
  \label{thm:common-div-linear-combo}
  \label{thm:linear-combo-divisors}
  \label{linear-combo-divisors}
  Any common divisor of $a$ and $b$ is a divisor of any linear
  combination thereof.
\end{lemma}
\begin{proof}
  Let $L = ax + by$ be a linear combination of $a$ and $b$. We have
  that $d \div a$, and therefore $a = dm$. We also have $d \div b$,
  and therefore $b = dn$. Therefore,

  \begin{rcl}
    L & = & (dm)x + (dn)y \\
      & = & d (mx + ny) \\
  \end{rcl}

  Therefore, $d \div L$.
\end{proof}

\begin{corollary}
  Given $a = qb + r$, $\gcdp{a,b} = \gcdp{b,r}$.
\end{corollary}
\begin{proof}
  $a = qb + r$ is a linear combination of $b$ and $r$. By
  \cref{thm:linear-combo-divisors}, all common divisors of $b$ and $r$
  also divide $a$.

  We also have $r = a - qb$, which implies that any common divisor of
  $a$ and $b$ is also a divisor of $r$.  This implies

  \begin{zz}
    \scomp{d \in \Z}{d \div a \land d \div b} = \scomp{d \in \Z}{d
      \div b \land d \div r}
  \end{zz}

  Because the two sets are equal, their maximum elements are
  equal. Therefore,

  \begin{zz}
    \gcdp{a,b} = \gcdp{b,r}
  \end{zz}

  QED.
\end{proof}

\begin{example}
  \label{ex:divisors-linear-combos}
  Remember from before,

  \begin{zz}
    42 = 1 \cdot 30 + 12
  \end{zz}

  The divisors of $42$ are

  \begin{zz}
    \mset{1,2,3,6,7,14,21,42}
  \end{zz}

  The divisors of $30$ are

  \begin{zz}
    \mset{1,2,3,5,6,12,30}
  \end{zz}

  The divisors of $r = 12$ are

  \begin{zz}
    \mset{1,2,3,6,12}
  \end{zz}

  The common divisors of all three are $\mset{1,2,3,6}$. Their
  greatest common divisor is $6$.

  We can write

  \begin{alignmath}{rcccl}
    42 & = & 1 \cdot 30 & + & 12 \\
    30 & = & 2 \cdot 12 & + & 6 \\
    12 & = & 2 \cdot 6 & + & 0 \\
  \end{alignmath}

  Therefore, $6 \div 12$, $6 \div 30$, and $6 \div 42$.
\end{example}

\subsection{Euclidean Algorithm}

The \term{Euclidean algorithm} is a fast method for finding the
greatest common divisor of two numbers. We purposefully stumbled upon
it in \cref{ex:divisors-linear-combos}. Choose positive integers
$a, b$

\begin{alignmath}{rclcrcl}
  a & = & q_1b + r_1 & \text{where} & 0 \le r_1 < b \\
  b & = & q_2r_1 + r_2 & \text{where} & 0 \le r_2 < r_1 \\
  r_1 & = & q_3r_2 + r_3 & \text{where} & 0 \le r_3 < r_2 \\
  r_2 & = & q_4r_3 + r_4 & \text{where} & 0 \le r_4 < r_3 \\
      &   &              & \dots \\
  r_{n - 2} & = & q_n r_{n - 1} + r_n & \text{where} & 0 \le r_n < r_{n - 1} \\
  r_{n - 1} & = & q_{n + 1}r_n + 0
\end{alignmath}

The last nonzero remainder, $r_n$, is the greatest common divisor.
The fact that the algorithm terminates follows from the fact that the
set $b > r_1 > r_2 > \dots > r_n$ must have a least element (by the
well-ordering principle), which implies $r_{n + 1} = 0$.

\begin{example}
  Let's use the Euclidean algorithm to find $\gcdp{541,130}$

  \begin{alignmath}{rcrcl}
    541 & = & 4 \cdot 130 & + & 21 \\
    130 & = & 6 \cdot 21 & + & 4 \\
    21 & = & 5 \cdot 4 & + & 1 \\
    4 & = & 4 \cdot 1 & + & 0 \\
  \end{alignmath}

  Since $1$ is the last nonzero remainder, it's the greatest common
  divisor.
\end{example}

\begin{remark}
  Aaron was tricky. $541$ is a prime number, which implies its only
  divisors are $1$ and $541$. Therefore, its greatest common divisor
  with any number absolutely less than $541$ will obviously be $1$.
\end{remark}

\begin{example}
  Let's find $\gcdp{10469,10013}$ using the Euclidean algorithm.

  \begin{alignmath}{rcrcl}
    10469 & = & 1 \cdot 10013 & + & 456 \\
    10013 & = & 21 \cdot 456 & + & 437 \\
    456 & = & 1 \cdot 437 & + & 19 \\
    437 & = & 23 \cdot 19 & + & 0 \\
  \end{alignmath}

  Therefore, $\gcdp{10469,10013} = 19$
\end{example}

\begin{example}
  Find $\gcdp{2717,2431}$ using the Euclidean algorithm

  \begin{alignmath}{rcrcl}
    2717 & = & 1 \cdot 2431 & + & 286 \\
         & \dots &
  \end{alignmath}
\end{example}

\begin{remark}
  Turns out, worst case, the Euclidean algorithm terminates in
  $5 \cdot \log_{10} r$ steps.
\end{remark}
