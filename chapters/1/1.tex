\section{Induction}

There are five axioms, called the \term{Peano
  axioms}\footnote{``Peano'' is pronounced ``pay ahh no''.}, which we
can use to define most properties of the natural numbers. We'll be
looking at the fifth axiom, called the ``inductive axiom''.

\begin{aside}
  There are loads of textbooks that prove very basic properties of the
  natural numbers using the Peano axioms. Such properties include
  things like

  \begin{enumerate}
  \item for all natural numbers $a, b, c$

    \begin{zz}
      a + (b + c) = (a + b) + c
    \end{zz}
  \end{enumerate}

  It's not very interesting, but it's great practice if you aren't
  used to proofs. If you are interested in the topic, I would
  recommend Edmund Landau's \xti{Foundations of Analysis}
  \cite{landau}, right after you visit your local mental hospital.
\end{aside}

\begin{definition}[Strong induction]
  \label{def:strong-induction}
  If there is a set $A$ such that
    \begin{itemize}
    \item $1 \in A$,
    \item for every $n$, $\mset{1, 2, \dots, n} \subeq A$ implies
      $S(n) \in A$,
    \end{itemize}

    then every natural number is in $A$.
\end{definition}

The last axiom is called ``strong induction''. There is a weaker, more
common version of the axiom, called ``weak induction''.

\begin{definition}[Weak induction]
  \label{def:weak-induction}
  If there is a set $A$ such that

  \begin{itemize}
  \item $1 \in A$
  \item $n \in A$ implies $S(n) \in A$
  \end{itemize}

  then every natural number $n$ is in $A$.
\end{definition}

Weak induction can be proven from strong induction
(\cref{exc:strong-weak}).\footnote{Generally, the terms ``strong X''
  and ``weak X'' mean that ``weak X'' can be proven from ``strong
  X''.} The concept is usually called \term{mathematical induction},
or just \term{induction}. In general, we apply this in proofs by
having some function $P$, which takes a natural number, and decides if
a given proposition is true or not. If $\scomp{n \in \N}{P(n)}$
satisfies the induction axiom, then we conclude $P(n)$ holds for all
$n$.

\begin{aside}
  Some logic systems don't have the dichotomy of ``true'' or
  ``false'', also called the \term{law of excluded middle}, which
  makes my explanation of induction somewhat reductionist.  If you are
  interested in the topic, you can read \link{John Halleck's index of
    logic
    systems}{http://www.cc.utah.edu/~nahaj/logic/structures/index.html}. \cite{halleck}
\end{aside}

\begin{definition}[Well-ordering principle]
  \label{def:well-ordering}
  \label{well-ordering}
  \label{least-element}
  Every non-empty subset of the natural numbers has a least element.
\end{definition}

\begin{remark}
  The well-ordering principle is equivalent to the principle of
  mathematical induction (\cref{exc:well-ordering}).
\end{remark}

\begin{lemma}
  \label{thm:n-positive-ints}
  The sum of the first $n$ odd positive integers is equal to
  $n^2$. That is,

  \begin{zz}
    n^2 = \sum_{k = 1}^n 2k - 1
  \end{zz}
\end{lemma}

\begin{proof}
  We have a satisfactory number of base cases

  \begin{rcl}
    1^2 & = & 1 \\
    2^2 & = & 1 + 3 \\
    3^2 & = & 1 + 3 + 5 \\
  \end{rcl}

  For our inductive case, we have that $P(n - 1)$ holds.

  \begin{rcl}
    (n - 1)^2 & = & 1 + 3 + 5 + \dots + \brackets{2(n - 1) - 1} \\
              & = & 1 + 3 + 5 + \dots + \parens{2n - 3} \\
  \end{rcl}

  We claim that 

  \begin{rcl}
    n^2 & = & 1 + 3 + 5 + \dots + \parens{2n - 1}
  \end{rcl}

  First note that this is equal to

  \begin{rcl}
    n^2 & = & \brackets{1 + 3 + 5 + \dots + \parens{2n - 3}} + \parens{2n - 1} \\
        & = & \parens{n - 1}^2 + \parens{2n - 1} \\
        & = & \parens{n^2 - 2n + 1} + \parens{2n - 1} \\
        & = & n^2
  \end{rcl}

  QED.
\end{proof}

\begin{lemma}
  For $x \in \R$, $x \ne 1$, and $n \in \N$

  \begin{zz}
    \frac{x^n - 1}{x - 1} = 1 + x + x^2 + \dots + x^{n - 1}
  \end{zz}
\end{lemma}

\begin{proof}
  In the base case, we have $\frac{x^1 - 1}{x - 1} = 1$.

  In the inductive case, we assume, for all $1 < k \le n - 1$

    \begin{zz}
      \frac{x^k - 1}{x - 1} = \sum_{j = 0}^{k - 1} x^j
    \end{zz}

    We have

    \begin{rcl}
      \sum_{j = 0}^{n - 1} x^j & = & \sum_{j = 0}^{n - 2} x^j + x^{n - 1} \\
                               & = & \frac{x^{n - 1} - 1}{x - 1} + \frac{x - 1}{x - 1}x^{n - 1} \\
                               & = & \frac{x^{n - 1} - 1}{x - 1} + \frac{x^n - x^{n - 1}}{x - 1} \\
                               & = & \frac{x^{n - 1} - 1}{x - 1} + \frac{x^n - x^{n - 1}}{x - 1} \\
                               & = & \frac{x^n - 1}{x - 1} \\
    \end{rcl}

    QED.
\end{proof}

\begin{corollary}
  If $a, b \in \N$, and $b \ge 2$, then $a < b^a$
\end{corollary}
\begin{proof}
  Note that $a = 1 + 1 + \dots + 1$, where $1$ appears $a$ times.

  We have that

  \begin{zz}
    1 + 1 + \dots + 1 < 1 + b + b^2 + \dots + b^{a - 1}
  \end{zz}

  By the previous lemma, the right-hand side of that inequality is
  $\frac{b^a - 1}{b - 1}$. We therefore have

  \begin{zz}
    \frac{b^a - 1}{b - 1} \le b^{a - 1} < b^a
  \end{zz}

  This implies $a < b^a$. QED.
\end{proof}