\section{Induction}

There are 5 axioms, called the \term{Peano axioms}\footnote{``Peano''
  is pronounced ``pay ahh no''.}, which we can use to define most
properties of the natural numbers.

\begin{aside}
  There are loads of textbooks that prove very basic properties of the
  natural numbers using the Peano axioms. Such properties include
  things like

  \begin{enumerate}
  \item for all natural numbers $a, b, c$

    \begin{zz}
      a + (b + c) = (a + b) + c
    \end{zz}
  \end{enumerate}

  It's not very interesting, but it's great practice if you aren't
  used to proofs. If you are interested in the topic, I would
  recommend Edmund Landau's \xti{Foundations of Analysis}
  \cite{landau}, right after you visit your local mental hospital.
\end{aside}

\begin{definition}[Strong induction]
  \label{def:strong-induction}
  If there is a set $A$ such that
    \begin{itemize}
    \item $1 \in A$,
    \item for every $n$, $\mset{1, 2, \dots, n} \subeq A$ implies
      $S(n) \in A$,
    \end{itemize}

    then every natural number is in $A$.
\end{definition}

The last axiom is called ``strong induction''. There is a weaker, more
common version of the axiom, called ``weak induction''.

\begin{definition}[Weak induction]
  \label{def:weak-induction}
  If there is a set $A$ such that

  \begin{itemize}
  \item $1 \in A$
  \item $n \in A$ implies $S(n) \in A$
  \end{itemize}

  then every natural number $n$ is in $A$.
\end{definition}

Weak induction can be proven from strong induction
(\cref{exc:strong-weak}).\footnote{Generally, the terms ``strong X''
  and ``weak X'' mean that ``weak X'' can be proven from ``strong
  X''.} The concept is usually called \term{mathematical induction},
or just \term{induction}. In general, we apply this in proofs by
having some function $P$, which takes a natural number, and decides if
a given proposition is true or not. If $\scomp{n \in \N}{P(n)}$
satisfies the induction axiom, then we conclude $P(n)$ holds for all
$n$.

\begin{aside}
  Some logic systems don't have the dichotomy of ``true'' or
  ``false'', also called the \term{law of excluded middle}, which
  makes my explanation of induction somewhat reductionist.  If you are
  interested in the topic, check out \link{John Halleck's Logic
    Systems}{http://www.cc.utah.edu/~nahaj/logic/structures/systems/index.html}. \cite{halleck}
\end{aside}

\begin{definition}[Well-ordering principle]
  Every non-empty subset of the natural numbers has a least element.
\end{definition}

\begin{remark}
  The well-ordering principle is equivalent to the principle of
  mathematical induction (\cref{exc:well-ordering}).
\end{remark}

\begin{lemma}
  The sum of the first $n$ positive integers is equal to $n^2$. That
  is,

  \begin{zz}
    n^2 = \sum_{i = 1}^n i
  \end{zz}

  Let's define $S$ to be the set of all natural numbers for which
  $P(n)$ does not hold.
\end{lemma}

% -------------------------------------------------------------------- %
\subsection*{Exercises}

\begin{ExerciseList}
  \Exercise{Prove that strong induction (\cref{def:strong-induction})
    implies weak induction (\cref{def:weak-induction}).}
  \label{exc:strong-weak}

  \Exercise{Using induction, prove that, for all natural numbers
    $n > 2$, there are no integer solutions to

    \begin{zz}
      x^n + y^n = z^n
    \end{zz}
  }
  \label{exc:fermat}

  \Exercise{Prove that the well-ordering principle
    (\cref{def:well-ordering}) is equivalent to mathematical
    induction.}
  \label{exc:well-ordering}
\end{ExerciseList}