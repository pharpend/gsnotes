\chapter{Introduction}

This is a collection of my notes, as well as elaborations on things
that confused me.

If you're reading this in a PDF reader, you can click on references,
and your PDF reader will take you to the appropriate place. Examples
are

\begin{itemize}
\item Citations: \cite{savin}
\item Footnotes\footnote{Hello}
\item References to other content, like \cref{ch:euclid}.
\item URLs, like \link{the course
    website}{http://www.math.utah.edu/~wood/4400}.
\end{itemize}

If you notice any errors, please email
\path{<peter.harpending@utah.edu>}.

This class assumes you know what proofs, sets, set comprehensions, and
propositions are.

\section{What is Number Theory?}

Number theory is the study of the \term{natural numbers}, usually
abbreviated as $\N$.

\begin{zz}
  \N = \mset{0,1,2,3,4,\dots}
\end{zz}

However, that quickly becomes rather limiting. So, we also study the
following sets:

\begin{enumerate}
\item The \term{integers}, usually denoted with a pretentious
  $\Z$. The `z' is short for ``zahlen'', which means ``numbers'' in
  German. \cite{w-integers}

  \begin{zz}
    \Z = \mset{\dots, -4, -3, -2, -1, 0, 1, 2, 3, 4, \dots}
  \end{zz}
\item The \term{rational numbers}, also called the \term{quotient
    numbers}, usually denoted with $\Q$.

  \begin{zz}
    \Q = \scomp{\frac{m}{n} \in \R}{m, n \in \Z;\; n \ne 0}
  \end{zz}

  The `q' stands for ``quotient''. ``Rational'', in this context,
  means ``can be expressed as a ratio'', not ``reasonable''. That
  said, the set of real numbers that are \xti{not} rational are called
  the ``irrational numbers''. Moreover, as we'll see, the rational
  numbers are considerably more reasonable than the irrational
  numbers.
\item We're subtly assuming (not entirely unreasonably) that you know
  what real numbers are. They are very difficult to define in a way
  that is not problematic. The real numbers are denoted $\R$. 
\item Finally, we have the \term{complex numbers}. These are numbers
  with two units, one of which is $i = \sqrt{-1}$.

  \begin{zz}
    \C = \scomp{a + bi}{a, b \in \R}
  \end{zz}
\end{enumerate}

\begin{aside}[Uniqueness of real numbers]
  You might be tempted to define

  \begin{zz}
    \R = \scomp{z + \sum_{p = 1}^{\infty} a_p \times 10^{-p}}{a_p \in
      \N, z \in \Z}
  \end{zz}

  This is a silly formal way of writing something like
  $-14.20487172\dots$ (an irrational number with infinite digits). The
  integer part, $z$, is $-14$. Then $a = \mset{2,0,4,8,7,\dots}$. In
  the summation notation, this would be

  \begin{zz}
    -14 + \parens{2 \times 10^{-1}} + \parens{0 \times 10^{-2}} +
    \dots
  \end{zz}

  which is equivalent to

  \begin{zz}
    -14 + 0.2 + 0.00 + 0.004 + 0.0008 + \dots
  \end{zz}

  The problem is when we have a number like

  \begin{zz}
    n = 0.99999\dots
  \end{zz}

  From the series definition, this is convergent to $1$, because for
  any $\epsilon > 0$, there is always an $n \in \N$ such that

  \begin{zz}
    \abs{1 - \sum_{p = 1}^{n} a_p \times 10^{-p}} < \epsilon
  \end{zz}

  In layman's terms, this means that, for any real number $\epsilon$,
  if we have a sufficient number of digits, we can get close enough to
  $1$ such that the difference between $1$ and $0.999\dots$ (with $n$
  digits) is less than $\epsilon$. This is the exact definition of
  convergence of an infinite series. \cite{taylor}

  There's a much more elegant---but harder to typeset---proof of this
  on \link{Peter Alfeld's
    site}{http://www.math.utah.edu/~pa/math/sets/unique.html}. \cite{pa-unique}

  The problem here is that \xtb{decimal representations of real
    numbers are not unique}, meaning that we can write down two
  different numbers that converge to the same number.

  That particular problem probably won't come up in our class too
  often. It is worth mentioning, though.

  The most common non-problematic method of defining real numbers is
  to use \term{Dedekind cuts}. If you would like to learn more about
  this, I suggest reading Edmund Landau's \xti{Foundations of
    Analysis}. \cite{landau}
\end{aside}
